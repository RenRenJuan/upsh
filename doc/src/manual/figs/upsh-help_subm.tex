\begin{verbatim}
usage: upsh {h,v}~ {<c>,s} {<m>,n} [b,e/1,f/1,d,p,o,r,w,a] - Program [Args]*
        h : help, overrides,
        v : version, overrides,
        b : be verbose,
        c : command line invocation (default),
        s : script invocation,
        e : errors file (default: std_out),
        f : name of functor for top goal (default: <filename>.pl then, main),
        d : working directory (default: Dir='.'),
        m : compile Program (default),
        n : consult Program,
        p : surpress loading info (of Program),
        o : donot translate, Arg, very=nice=tara to very(nice(tara)),
        r : be strict about the `-' .
        w : wait for input at end of execution.
        a : last argument holds all read variable lists (upsh_vs/1).

For scripts add following to a prolog file, and do chmod u+x <PlFile>
#! /bin/sh
exec upsh s - $0 "$@"
\end{verbatim}
