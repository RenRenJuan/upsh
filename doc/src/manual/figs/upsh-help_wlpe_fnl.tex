\begin{verbatim}
usage: upsh {h,v}~ {(c),s} {(m),n} [b,e/1,f/1,d/1,p,o,r,w,a] - Program [Args]*.
        h : help; overrides,
        v : version; overrides,
        c : command line invocation (default),
        s : script invocation,
        m : compile Program (default),
        n : consult Program,
        b : be verbose,
        e : errors file (default: std_out stream),
        f : name of functor for top goal (default: <Program>/1 then main/1),
        d : working directory (default: '.'),
        p : surpress loading info (when loading Program),
        o : do not translate Args; e.g. very=nice=tara not to very(nice(tara)),
        r : be strict about the '-',
        w : wait for input at end of execution,
        a : last term will hold list of all read variables (upsh_vs/1).

For scripts add the following lines to the source and do chmod u+x <Program>.pl
#!/bin/sh
exec upsh s - $0 "$@"
\end{verbatim}
